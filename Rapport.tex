  
\documentclass[12pt]{article}
\usepackage[francais]{babel}
\usepackage{natbib}
\usepackage{url}
\usepackage[utf8x]{inputenc}
\usepackage{graphicx}
\graphicspath{{images/}}
\usepackage{parskip}
\usepackage{fancyhdr}
\usepackage{vmargin}
\usepackage{xcolor}
\usepackage{bbm}
\usepackage{amsmath,amssymb}
\usepackage{amsthm}


\title{Protocole de communication basé TCP sécurisé par RSA}
\author{PIARD A. - JACQUET R.}
\date{\today}

\makeatletter
\let\thetitle\@title
\let\theauthor\@author
\let\thedate\@date
\makeatother

\pagestyle{fancy}
\fancyhf{}
\rhead{\footnotesize{\theauthor}}
\lhead{\footnotesize{\thetitle}}
\cfoot{\thepage}
\def\dotfill#1{\cleaders\hbox to #1{.}\hfill}
\newcommand\dotline[2][.5em]{\leavevmode\hbox to #2{\dotfill{#1}\hfil}}

\newcommand{\M}{\mathbbm{M}}
\newcommand{\N}{\mathbbm{N}}
\newcommand{\Z}{\mathbbm{Z}}
\newcommand{\Q}{\mathbbm{Q}}
\newcommand{\R}{\mathbbm{R}}
\newcommand{\C}{\mathbbm{C}}
\newcommand{\G}{\mathbbm{G}}
\newcommand{\K}{\mathbbm{K}}
\newcommand{\F}{\mathbbm{F}}

\newtheorem{theorem}{Théorème}
\newtheorem{corollaire}{Corollaire}
\newtheorem{lemma}{Lemme}
\newtheorem{prop}{Proposition}
\theoremstyle{definition}
\newtheorem{definition}{Définition}
\newtheorem{example}{Exemple}
\newtheorem*{examples}{Exemples}
\newtheorem{exo}{Exercice}	

\begin{document}

%%%%%%%%%%%%%%%%%%%%%%%%%%%%%%%%%%%%%%%%%%%%%%%%%%%%%%%%%%%%%%%%%%%%%%%%%%%%%%%%%%%%%%%%%

\begin{titlepage}
	\centering
    \vspace*{0.5 cm}
    \textsc{\LARGE Projet Réseaux et Systèmes, RSA . 2020-2021}\\[1.0 cm]
    \dotline[15pt]{15cm}\\
	\includegraphics[scale = 2.2]{logo.png}
	\dotline[15pt]{15cm}\\
	\vspace{1.5cm}
	\textsc{\Large Faculté des Sciences et Techniques}\\
	\textsc{\large Master 1 - Maths. CRYPTIS}\\[1.0 cm]
	\rule{\linewidth}{0.2 mm} \\[0.4 cm]
	{ \huge \bfseries \color{blue} \thetitle}\\
	\rule{\linewidth}{0.2 mm} \\[1.5 cm]
	
	\begin{minipage}{0.4\textwidth}
		\begin{flushleft} \large
			\emph{A l'attention de :}\\
			M. BROS Maxime\\
			\phantom{a}\\
			\phantom{a}\\
		\end{flushleft}
	\end{minipage}
	\begin{minipage}{0.5\textwidth}
    	\begin{flushright} \large
		\emph{Rédigé par :}\\
		PIARD A.\\
		JACQUET R.\\
		\end{flushright}
	\end{minipage}\\[2 cm]
\end{titlepage}
%%%%%%%%%%%%%%%%%%%%%%%%%%%%%%%%%%%%%%%%%%%%%%%%%%%%%%%%%%%%%%%%%%%%%%%%%%%%%%%%%%%%%%%%%

\tableofcontents
\pagebreak

%%%%%%%%%%%%%%%%%%%%%%%%%%%%%%%%%%%%%%%%%%%%%%%%%%%%%%%%%%%%%%%%%%%%%%%%%%%%%%%%%%%%%%%%%
\section*{Introduction}
Le cryptosystème RSA doit son nom aux initiales de ses 3 créateurs, à savoir Rivest (Ron), Shamir (Adi) et Adleman (Leonard). C'est un algorithme de cryptographie asymétrique inventé en 1977. Ce cryptosystème date par conséquent de plus de 40 ans mais est encore utilisé dans certains cas.\\\\
A l'époque il était très efficace car les ordinateurs étaient bien moins puissants qu'aujourd'hui, donc des tailles de clés de 256 bits étaient suffisantes pour chiffrer. Aujourd'hui, des tailles de clés de 1024 bits sont nécessaires de manière à ce que le cassage des clefs pour assurer un déchiffrement soit très coûteux, principalement en temps. D'autres méthodes de chiffrement comme l'AES, Advanced Encryption Standard, sont bien plus efficaces aujourd'hui. Mais l'AES est un chiffrement symétrique.\\\\
Il est donc fréquent d'utiliser RSA pour échanger des clefs de chiffrement symétriques (pour AES par exemple) de manière sécurisée, puis de s'échanger des informations à travers ce chiffrement symétrique.
\vfill \eject

\section{Fonctions et problèmes rencontrés}
\vspace{12pt}
	\subsection{La fonction creationNombreTaille(n)}
	Le paramètre $n$ est un entier en base $10$ qui indiquera que le nombre créé doit être composé de $n$ chiffres.\\
	Etant donné que nous cherchons des nombres premiers, dans cette fonction seuls des nombres impairs sont créés. C'est la fonction suivante qui déterminera si un nombre est premier ou non.	
	\subsection{La fonction premier(n)}
	Cette fonction, grâce à la commande "openssl prime " va nous dire si le paramètre $n$, qui est un nombre, est premier ou non. Si ce nombre n'est pas premier, le but va être de le modifier tout en utilisant le plus possible les chiffres qui le composent. Comme expliqué dans le sujet. \\
C'est une expression régulière qui va récupérer l'information de si le nombre est ou non premier. Cette expression régulière va rechercher "is not prime" dans la sortie de la commande \textit{openssl prime n}. Ici, rechercher "not" est suffisant. Si l'expression régulière ne match pas, cela veut dire que le nombre est premier donc la fonction va renvoyer ce nombre. Si l'expression régulière match, cela veut dire que le nombre n'est pas premier et qu'il faut le changer puis tester sa primalité.\\
A ce moment là on le modifie en gardant le plus possible sa forme précédente. On avait ici soulevé un problème.\\\\
\textbf{Problème : }Si le nombre testé, qui est non premier, est par exemple de cette forme : $50421$, c'est à dire que son deuxième chiffre en partant de la gauche, est un $0$, alors suite à la modification de ce nombre, on obtiendra $0421X$, où $X \in \lbrace 1,3,7,9 \rbrace$. C'est à dire $421X$. Or, ce nombre est composé de 4 chiffres. Dans le cas général, de $n-1$ chiffres. Donc il n'est plus de la taille souhaitée. \\\\
\textbf{Solution : }On a rajouté une condition lors de la modification de $n$. Si le deuxième chiffre est un $0$, on en choisit au hasard un nouveau, compris entre $1$ et $9$. Ainsi, le nombre voulu sera composé de $n$ chiffres. On test alors la primalité de ce nouveau nombre, de la même manière que la première fois et on répète cette opération tant qu'on ne tombe pas sur un nombre premier.\\\\
De plus, lors de la modification du nombre, s'il n'est pas premier, il faut changer l'avant dernier chiffre car celui-ci sera égal au dernier chiffre du précédent nombre. Il sera donc égal soit à $1$, soit à $3$, soit à $7$ ou soit à $9$. Or, Les chiffres qui composent le nombre, hormis le premier et le dernier chiffre, doivent être compris entre $0$ et $9$. Donc on remplacera celui-ci par un chiffre compris entre $0$ et $9$.

%Rajouter des screens, cette fonction est intéressante
	\subsection{Les fonctions egcd(a,b) et modinv(a,m)}
	Ces deux fonctions sont liées. La fonction egcd renvoie le pgcd des nombres $a$ et $b$ ainsi que les coefficient de Bézout de $a$ et $b$. Cette fonction correspond donc à l'algorithme d'Euclide étendu.\\
	Quant à elle, la fonction modinv(a,m), renommée en inverseModulo(a,m) dans notre programme, fait appel à la fonction egcd et renvoie l'inverse modulaire d'un nombre.\\
	On utilisera donc ces fonctions pour calculer $d=e^{-1}  (mod(\phi (n)))$ avec $\phi(n) = (p-1) \times (q-1)$ et $n = p \times q$.
	 \subsection{La fonction lpowmod(x,y,n)}
Cette fonction a été renommée powmod(x,y,n) dans notre programme.\\
	 Comme indiqué dans le sujet, cette fonction permet de calculer $x^y$ $mod(n)$ bien plus rapidement qu'avec l'opération d'exponentiation ** de python, suivi de la réduction modulo n du résultat obtenu.\\
	 Cette fonction sera utilisée pour chiffrer et déchiffrer par RSA. En effet, l'opération de chiffrement (respectivement de déchiffrement) pour RSA est $m^e$ $mod(n)$ (respectivement $c^d$ $mod(n)$ ).\\
	 Etant sujets à pratiquer RSA avec de grands nombres premiers, l'implémentation de cette fonction est primordiale pour optimiser la vitesse des calculs.
	 \subsection{La fonction splitParam(str,size)}
	
\vfill \eject
\section{Jeu d'essais}


\end{document}
